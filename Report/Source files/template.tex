
\section{References}
Here's a reference to an article \cite{universalFER}.

Here's a reference to a Website \cite{insat}.

\section{Lists}
\subsection{Bullet lists}
Here's an example of a bullet list:
\begin{itemize}
\item Idea 1.
\item Idea 2.
\end{itemize}
\medskip \par

\subsection{Enumerated lists}
Here's an example of an enumerated list:
\begin{enumerate}
\item Idea 1 ;
\item Idea 2.
\end{enumerate}
\medskip \par

\section{Figures}
Figure \ref{fig:exemple1} is an example of a figure.
\begin{figure}[!h]
\centering
\includegraphics[width=0.3\textwidth]{images/INSAT.jpg}
\caption{This is an example of a figure.}
\label{fig:exemple1}
\end{figure}

Figure \ref{fig:exemple2} is an example of insertion of many images (\ref{fig:exemple21} and \ref{fig:exemple22}) in one figure.
\begin{figure}[!h]
\begin{center}
\subfloat[]{\includegraphics[width=0.2\textwidth]{images/INSAT.jpg}\label{fig:exemple21}}\quad 
\subfloat[]{\includegraphics[width=0.3\textwidth]{images/INSAT.jpg}\label{fig:exemple22}}
\caption{This is an example of a multi-figure: (a) small sub-figure and (b) large sub-figure.}
\label{fig:exemple2}
\end{center}
\end{figure}

\section{Tables}
Table \ref{tab:exemple1} is an example of a table with fixed columns.

\begin{table}[!h]
\caption{This is an example of a table} 
\label{tab:exemple1}
\centering
\renewcommand{\arraystretch}{1.65}
\begin{tabular}{| C{.11\textwidth} || C{.35\textwidth} | p{.45\textwidth} |} 
\hline
\textbf{a} & \textbf{b} & \textbf{c} \\ 
\hline 
\hline 
1 & \nohyphens{No hyphens avoids cutting the words.} & \nohyphens{...} \\ 
\hline 
2 & \nohyphens{...} & \nohyphens{...} \\ 
\hline 
\end{tabular}
\end{table}

Table \ref{tab:exemple2} is an example of a table where columns' widths adapt to the content.

\begin{table}[!h]
\caption{This is an example of a table with multilines et multicolumns} 
\label{tab:exemple2}
\centering
\renewcommand{\arraystretch}{1.65}
\begin{tabular}{|l||c|c|c|}
\cline{2-4}
\multicolumn{1}{c|}{\nohyphens{}} & a & b & c\\
\hline
\hline
1 & \nohyphens{...} & \nohyphens{...} & \nohyphens{...}\\ 
\hline 
2 & \multicolumn{2}{c|}{\nohyphens{...}} & \nohyphens{...}\\ 
\hline 
\multirow{2}*{3} & ... & ... & ... \\
\cline{2-4}
& \nohyphens{...} & \nohyphens{...} & \nohyphens{...}\\ 
\hline 
\end{tabular}
\end{table}

\section{Algorithms}
Algorithm \ref{algo:exemple} is an example. 
\begin{algorithm}[!h]
\setstretch{1.15}
 \caption{Process}
 \label{algo:exemple}
 \SetKwInOut{Input}{Input}
 \SetKwInOut{Output}{Output}
 \Input{$X= \{x_{i} \in\mathbb{R}^{d}\}_{i=1}^{n}$ : data matrix dimensioned $d \times n $\\
$Y = \{y_{i} \in\{1, ..., c\}\}_{i=1}^{n}$ : vector of $n$ labels\\}
\Output{$T$ : processed matrix}

... $\gets$ ...\;
\For{$i\gets 1$ \KwTo $n$ }{
\eIf{\textsf{...}}
 {
 ...\;
 }
 {
 \eIf{\textsf{\upshape ...}}
 {
 ...\;
 }
 {
 ...\;
 }
 ...
 }				 
 }
 $T \gets$ ...
\end{algorithm}

\section*{Conclusion}

\chapter*{Conclusion and perspectives}
\addcontentsline{toc}{chapter}{Conclusion and perspectives}
\markboth{Conclusion and perspectives}{}

\begin{appendix}
\chapter{Appendix 1}
Insert your appendixes here if you need.
\end{appendix}

%\spacing{1}
\bibliographystyle{unsrt}
\bibliography{references}
\end{document}
\subsubsection{Micro expressions}
Since facial expressions appear spontaneously and unconsciously; they can reveal hidden feelings and thoughts.
They appear quickly before a person is fully aware which makes it challenging to control and hide an emotional reaction. Micro-expressions are the rapid display of a concealed emotion; occurring within the duration of 1/5 and 1/25 seconds, compared to ½ to 4 seconds for a macro-expression.
Recognizing them provides an insight into true thoughts and emotions.

\subsubsection{DenseNet}
A DenseNet is a sort of convolutional neural network that utilizes dense 
connections among layers, via Dense Blocks, wherein we connect all layers (with 
matching feature-map sizes) with each other. To maintain the feed-forward 
nature, each layer obtains extra inputs from all previous layers and passes on its 
personal feature maps to all subsequent layers.
\autoref{fig:densenet}
\begin{figure}[H]
    \centering
    \includegraphics[width=0.7\textwidth]{figures/A-visualization-of-the-DenseNet-201-architecture-Each-layer-takes-all-preceding.png}
    \caption{A visualization of the DenseNet-201 architecture}
    \label{fig:densenet}
    \cite{denseNet}
\end{figure}
DenseNet-201 trained on ImageNet dataset,this model achieves 77.42%top-1 and 
93.66\% top-5 accuracy.


\\
Micro-expression analysis is a key to detecting lies and deceit. In fact, when the content of a speech contradicts the emotion expressed on the face, it means that the speaker is concealing what he truly thinks about a matter. 

Trained on ImageNet Large Scale Visual Recognition Challenge 2012 classification dataset, consisting of 1.2 million training images, with 1,000 classes of objects; this model achieves 75.2\% top-1 and 92.5\% top-5 accuracy.

Trained on ImageNet Large Scale Visual Recognition Challenge 2012 classification dataset, consisting of 1.2 million training images, with 1,000 classes of objects; this model achieves 87\% top-1 and 96.3\% top-5 accuracy.

\begin{figure}[H]
    \centering
    \includegraphics[width=0.5\textwidth]{figures/State of art/morphcast.PNG}
    \caption{Morphcast editing phase}
    \label{fig:morphcast}
    \cite{morphcast}
\end{figure}
\begin{figure}[H]
    \centering
    \includegraphics[width=0.4\textwidth]{figures/State of art/morphcast_test.png}
    \caption{Morphcast emotional tracking}
    \label{fig:morphcast_test}
\end{figure}

\begin{figure}[H]
    \centering
    \includegraphics[width=0.3\textwidth]{figures/google colab logo.png}
    \caption{Google Colab's logo}
    \label{googleColab}
\end{figure}
Now we will compare the results to choose the the best model and optimize its hyper-parameters.

\chapter*{\textbf{List of Acronyms}}
%\addcontentsline{toc}{chapter}{List of Acronyms}
\markboth{List of acronyms}{}
\begin{itemize}
\item \textbf{FER} Facial Expression Recognition
\item \textbf{CNN} Convolutional Neural Network
\item \textbf{FC} Fully Connected
\end{itemize}